This section presents the experiment that has been carried out and with two main objectives (1)  evaluating the understandability, applicability and usability of the methodological contributions; and (2) evaluating the usability of the Prot{\'e}g{\'e}LOV plug in.
\vspace{1mm}
\subsection{Experiment Settings}
For this experiment, fifteen ontology engineers built a small ontology by following the methodological guidelines provided here\footnote{\url{http://labs.mondeca.com/protolov/method.html}} and using the Prot{\'e}g{\'e}LOV plugin\footnote{\url{http://labs.mondeca.com/protolov/}}. After the interaction with the methodological guidelines and the plugin the participants filled in a questionnaire available here \footnote{\url{https://goo.gl/vMzHU2}}.
\vspace{1mm}
\subsection{Finding and observations}
For this preliminary evaluation we keep the number of evaluators low. The results indicate that while the tool needs some minor improvements, the majority of the evaluators still consider the tool to be useful. Moreover, most of the participants provided very good feedback about the proposed methodological guidelines.


%Finally, it is worth mentioning we have presented our plugin as demo papperused two pools for evaluating the tools: 16 experts in ontology modeling selected randomly and spontaneous visitors that came to our booth during the last ESWC 2015 in Portoroz. It has been difficult to receive the 
 

\begin{comment}
\subsection{Other evaluation?}

We have conducted an initial user driven evaluation of the tool. For this preliminary evaluation we keep the number of evaluators low; seven ontology engineers used our plugin and gave us an insight on their experience by filling in a questionnaire \footnote{\url{http://goo.gl/H4YgBJ}}. The results indicate that while the tool needs some minor improvements, the majority of the evaluators still consider the tool to be useful.
\end{comment}
\vspace{3mm}
\section{Conclusions and Future Work}
\label{sec:conclusions}
In this paper we have presented (1) initial guidelines that describe how to reuse available vocabularies at fine-grained level, i.e., by reusing specific classes and properties, and (2) a tool that provides technological support by means of a plugin for \protege that access LOV API. Moreover, we have presented the first evaluation we have performed for our guidelines and their technological support.


As future work we plan to evaluate the guidelines and tool with a large set of users. Also, we plan to include access to other respoitories, e.g., Bioportal and Bartoc. Finally, we plan to develop a plugin with the same functionalities for other tools, such as  Web-\protege\footnote{\url{http://webprotege.stanford.edu/}} and TopBraid composer\footnote{\url{http://www.topquadrant.com/tools/ide-topbraid-composer-maestro-edition/}}.


%During the demonstration we will show to potential users how to install the plugin and how to quickly build ontologies by reusing terms from LOV repository in \protege.

