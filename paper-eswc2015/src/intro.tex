So far, Linked Data principles and practices are being adopted by an increasing number of data providers, getting as result a global data space on the Web containing hundreds of LOD datasets \cite{Heath_Bizer_2011}. There are already several guidelines for generating, publishing, interlinking, and consuming Linked Data \cite{}. An important task, within the generation process, is to build the vocabulary to be used for modelling the domain of the data sources, and the common recommendation is to reuse as much as possible available vocabularies \cite{}. This reuse approach speeds up the vocabulary development, and therefore, publishers will save time, efforts, and resources. 

At the time of writing we have not found specific and detailed guidelines that describe how to reuse available vocabularies when building vocabularies. There are research efforts, like the NeOn Methodology \cite{}, and the W3C Working Group Note \cite{}, but they do not provide guidelines on how to reuse vocabularies at low fine grained, i.e., reusing specific classes and properties. Our claim is that this difficulty in how to reuse vocabularies at low fine grained is one of major barriers to the vocabulary development and in consequence to deployment of Linked Data.

In this paper we propose an initial set of guidelines for this task, and provide technological support by means of a plugin for Prot{\'e}g{\'e}, which is one the popular framework for developing ontologies in a variety of formats including OWL, RDF(S), and XML Schema. It is backed by a strong community of developers and users in many domains. One success on Prot{\'e}g{\'e} also lies on the availability to extend the core framework adding new functionalities by means of plug-ins.

At the same time, the recent success of Linked Open Vocabularies (LOV\footnote{\url{http://lov.okfn.org/dataset/lov/}}) as a central point for curated catalog of ontologies is helping to convey on best practices to publish vocabularies on the Web, as well as to help in the Data publication activity on the Web. LOV comes with many features, such as an API, a search function and a SPARQL endpoint.
 
Moreover, we propose in this paper to explore, design and implement a plug-in of LOV in Prot{\'e}g{\'e} for easing the development of ontologies by reusing existing vocabularies at low fine grained level. The tool has to improve the modeling and reuse of ontologies used in the LOD cloud, by providing the following features in Prot{\'e}g{\'e}:

\begin{itemize}
\item Import easily vocabularies from LOV into Prot{\'e}g{\'e}. 
\item Propose to the user a list of candidate vocabularies in LOV matching the term
\item Have an updating mechanism (synchronization) to LOV catalog
\item Check if a new vocabulary created in Prot{\'e}g{\'e} satisfied the LOV recommendations

\item Suggest to LOV a new created vocabulary within Prot{\'e}g{\'e}.
\end{itemize}

This paper presents Prot{\'e}g{\'e}, a first implementation of the LOV realized as a plugin for the ontology editor Prot{\'e}g{\'e}. 
 
\section{Methodological discussion? - based on Antoine discussions}
The activity of building vocabularies by reusing available vocabularies consits of
\begin{itemize}
	\item Search for suitable vocabularies to reuse. As we aforementioned, LOV is the current reference for vocabularies in the LOD cloud. We conduct this search using the terms of the domain of the data sources ... 
	\item Once we have selected the particular term, of a particular vocabulary we can take several approaches depending on the use case
	\begin{enumerate}
		\item Import the ontologies that contain the terms that you need. is the most powerful: it makes all the terms and all the axioms from the imported ontology part of your ontology. You can then reuse the terms, extend them with more axioms, link them to your own terms etc. The knowledge of the imported ontologies is integrated with the knowledge of your ontology and you can do rich inferences. However, if the imported ontology is very big and expressive and you just need a few terms, it's not always the best option.
		\item Reuse the terms that you need from external ontologies, but do not explicitly import them. In this you don't benefit from the axioms of the external ontology unless you copy the relevant axioms too. But you benefit from the fact that you are reusing existing terms that are possible well known. Take the example of FOAF. If you are using foaf:Person as a class for persons, even if you don't import the FOAF ontology, people will be able to use tools that are specifically made for FOAF. They will be able to query your dataset with the same SPARQL query as for other datasets using FOAF. Moreover, in case a more accurate definition of the term is needed, one can also use the "follow your nose" method which consists in looking up the URI of the term with an appropriate protocol, and getting back the ontology document. By doing so, you don't even need owl:imports very much. If you insist on having a valid OWL DL ontology, you can declare the terms by explicitly saying whether they are object properties, datatype properties, annotation properties, classes or individuals.
		\item Redefine your own terms. In this case you have full control of the terms. This may be a good choice if there is only poorly designed ontologies for the terms you need. Or, you may disagree with the definitions of the other ontologies. Or you are concerned by the fact that external ontologies may change at any time, possibly disappearing completely. In any case, if you have your own terms that mirror existing terms from other ontologies, you can also provide the correspondences between your terms and the others. In that case, I recommend that you provide those outside your ontology.		
	\end{enumerate}
		You can combine all the approaches, importing some ontologies (e.g., FOAF), reusing terms from others (e.g., Dublin Core) and making your own terms in some cases. The approach you choose in each case can be based on reasoning issues (expressiveness, size, conformance to DL, modularity), querying issues, robustness, linkage, reusability, taste, belief, mood, magnetic field, humour, weather, topology, spirituality, life :)
\end{itemize}


