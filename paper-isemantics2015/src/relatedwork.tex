%Show here some work related to plugin for helping in ontology development:Neon and other
%\todo{target to guidelines}
%\todo{I suggest to move this related work at the end}
In the literature there exist many attempts to advise vocabulary publishers on the importance of reusing terms, as indicated in \cite{janowicz2014five,jimenez2008}. However, to the best of our knowledge there are not guidelines to help vocabulary practitioners to reuse vocabularies in real-world scenario, and considering specific ontology/vocabulary elements. 

In the W3C Government Linked Data Best Practices document \cite{hyland14}, reusing vocabularies is recommended by providing to stakeholders a basic checklist when using or extending a vocabulary. It gives general guidance to follow before publishing the vocabulary, not guidance during the creation of the vocabulary. Our proposal is to guide the users during the implementation process.

Recently, Janowicz et al. have propopsed a five stars rating for Linked (Open) Data vocabulary use to ``encourgage data owners, engineers and practitioners to publish and use vocabularies on the Web'' \cite{janowicz2014five}. They make it clear that the rating do not refer to the qualitiy of the vocabularies. In the definition of the rating system, the third star is given to a vocabulary linked to other vocabularies by means of explicit alignments and import of external vocabularies. Our guidelines make it easier to vocabulary publishers to obtain at least three-star vocabularies.

With respect to our technological support, other initiatives similar to the tool we have developed can be found in the literature but not currently maintained. The BioPortal Reference Plugin\footnote{\url{http://protegewiki.stanford.edu/wiki/BioPortal_Reference_Plugin}} allows the user to insert into the ontology references to external ontologies and terminologies stored in BioPortal\footnote{\url{http://bioportal.bioontology.org/}}. The plugin allows to generate external reference of a selected term. Additionally, the BioPortal Import Plugin\footnote{\url{http://protegewiki.stanford.edu/wiki/BioPortal_Import_Plugin}} allows users to import classes from external ontologies stored in the BioPortal ontology repository. The user can import entire trees of classes with a desired depth and choose which properties to import for each class. However, those plugins work only with \protege 3.x releases and are not ported yet to recent versions. 

Most closely related to the {\protege}LOV plugin are approaches that use semantic search engine to support the process of editing an ontology and make large scale knowledge reuse automatically integrated in the tool. An example is the Watson Plugin \cite{neonguide2008} for the \neon Toolkit \cite{haase2008neon}, a plugin supporting the \neon life-cycle management using the Watson \cite{d2007watson} APIs\footnote{\url{http://watson.kmi.open.ac.uk/WS_and_API.html}}. However the similar plugin for Protege\footnote{\url{http://protegewiki.stanford.edu/wiki/Watson_Search_Preview}} is just a proof of concept rather than a real plugin.

 
